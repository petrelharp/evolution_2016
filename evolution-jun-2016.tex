\documentclass{beamer}
\usepackage{colortbl}
\usepackage{graphicx}
\usepackage{ulem}

\include{slide-macros}

\newcommand{\basedir}{files}

%%% macros
\def\migrate{\lambda_{\text{mig}}}
\def\mutrate{\lambda_{\text{mut}}}
\def\Tmig{T_{\text{mig}}}
\def\Tmut{T_{\text{mut}}}


\setbeamertemplate{blocks}[default]
\usecolortheme{rose}

\mode<presentation>
{
  % \usetheme{default}
  \usetheme{boxes}
  % or ...
  \usefonttheme[options]{structuresmallcapsserif}

  \setbeamercovered{transparent}
  % or whatever (possibly just delete it)
}


\usepackage[english]{babel}
% or whatever

\usepackage[latin1]{inputenc}
% or whatever

\usepackage{times}
\usepackage[T1]{fontenc}
% Or whatever. Note that the encoding and the font should match. If T1
% does not look nice, try deleting the line with the fontenc.


\title % (optional, use only with long paper titles)
{ Geography, and adaptation }

\author % (optional, use only with lots of authors)
{Peter Ralph}
% - Use the \inst{?} command only if the authors have different
%   affiliation.

\institute[UO]
{
    University of Oregon \\ Biology \& Mathematics
}
% - Use the \inst command only if there are several affiliations.
% - Keep it simple, no one is interested in your street address.

\date % (optional)
{Evolution! Austin, June 18, 2016}

\def\Put(#1,#2)#3{\leavevmode\makebox(0,0){\put(#1,#2){#3}}}


\begin{document}

\begin{frame}
  \titlepage
\end{frame}

%%%%%%%%%% %%%%%%%%%%%%% %%%%%%%%%%

\begin{frame}{}

    collaborators

\end{frame}

\begin{frame}{the Mojave Desert Tortoise: \only<1>{ \it Gopherus agassizii}}
    \begin{center}
  \includegraphics<1>[width=\textwidth]{\basedir/tortoise-in-burrow}
  \includegraphics<2>[width=.9\textwidth]{\basedir/range-abundance-map}
  \includegraphics<3>[height=.8\textheight]{\basedir/ivanpah-opens}
  \includegraphics<4>[height=.8\textheight]{\basedir/latimes-torts-delay-solar}
    \end{center}
\end{frame}

%%%%%%%%
\begin{frame}{The question}
  \begin{columns}[c]
    \begin{column}{.4\textwidth}

      {\Large How will changes to the landscape change gene flow? }


    \end{column}
    \begin{column}{.6\textwidth}
      \begin{center}

        \vfill
        \includegraphics[width=\textwidth]{\basedir/drecp-pref-alt-snapshot}
        \vfill

          % % \vspace{-4em}
          % \includegraphics[width=.7\textwidth]{\basedir/hagerty-2011-habitat-model}
          % \figcaption{Habitat predicted by 12-variable model, with least-cost paths,
          %     from 20 microsats.
          % {(Hagerty, Nussear, Esque, \& Tracy 2011)}}

      \end{center}
    \end{column}
  \end{columns}
\end{frame}


%%%%%%%%
\begin{frame}{Data collection}
  \begin{columns}
    \begin{column}{.5\textwidth}

      Thanks to
      {\newthing \bf lots of hard work},
      we have
      \begin{itemize}
        \item 83 GIS layers at 30m -- Jannet Vu (UCLA)
        \item<2-> tissue samples -- Dick Tracy, Chava Weitzman (UNR), Fran Sandmeier (U.\ Lindenwood)
        \item<3-> 270 tortoises: whole-genome sequences, average {\newthing 1.5x coverage}, ${}>10^{12}$ bp -- Evan McCartney-Melstad (UCLA)
          % \begin{itemize}
          %   \item<4-> mapped to Galapagos Tortoise
          % \end{itemize}
      \end{itemize}

    \end{column}
    \begin{column}{.5\textwidth}
      \begin{center}

          \includegraphics<1>[width=\textwidth]{\basedir/raster-list}
          \includegraphics<2>[width=\textwidth]{\basedir/fieldwork}
          \includegraphics<3>[width=2\textwidth]{\basedir/sample_map}
          % \includegraphics<4>[width=\textwidth]{\basedir/tortCoverages-cropped}

      \end{center}
    \end{column}
  \end{columns}
\end{frame}


%%%%%%%%
\begin{frame}{Pairwise divergence}
  \centering
        % made in tortoisescape/visualization/talk-plots.R
        \includegraphics<1>[width=\textwidth]{\basedir/everyone-pwp}
        \includegraphics<2>[width=\textwidth]{\basedir/pwp_etort-191}
        \includegraphics<3>[width=\textwidth]{\basedir/pwp_etort-57}
        \includegraphics<4>[width=\textwidth]{\basedir/pwp_etort-35}
        \includegraphics<5>[width=\textwidth]{\basedir/pwp_etort-285}
        \includegraphics<6>[width=\textwidth]{\basedir/pwp_etort-229}
        \includegraphics<7>[width=\textwidth]{\basedir/pwp_etort-273}
        \includegraphics<8>[width=\textwidth]{\basedir/pwp_etort-240}
   \vspace{2em}

  divergence $=$ (mean density of nucelotide differences)
\end{frame}

%%%%%%%
\begin{frame}{Descriptive statistics}
  \begin{columns}
    \begin{column}{.5\textwidth}
      \centering
        % made in tortoisescape/visualization/talk-plots.R
      \includegraphics<1-2>[height=0.9\textheight]{\basedir/everyone-pwp-vertical}
      \includegraphics<3->[width=\textwidth]{\basedir/IBD-spectrum-by-NS}
    \end{column}
    \begin{column}{.5\textwidth}
      \begin{itemize}
          \item Sequencing error $<.001$ (from mitochondria)
          \item Mean time to most recent common ancestor: 
              {\newthing $\approx 100,000$ generations}
              {\aside (calibrated to painted turtle)}
          \item<2-> Additional separation across north/south break:
              {\newthing $\approx 10\%$}
          \item<3-> Similar demography in north and south?
      \end{itemize}
    \end{column}
  \end{columns}

\end{frame}


%%%%%%%
\begin{frame}{Following a lineage}
  \begin{columns}
    \begin{column}{.5\textwidth}
      \centering
      \includegraphics{\basedir/lineages-hitting-time-onelineage}
    \end{column}
    \begin{column}{.5\textwidth}
      A {\newthing single lineage} back through time: \\
      \begin{itemize}
          \item jump rate is mean age of parent at birth
          \item tends to move towards regions that produce more offspring
      \end{itemize}


    \end{column}
  \end{columns}
\end{frame}

%%%%%%%
\begin{frame}{Aside: inverse problems}

\end{frame}

%%%%%%%
\begin{frame}{Lineage movement}

  Recall that we have {\newthing landscape variables}, e.g.,
  \begin{align*}
    g_1(x) &= ( \text{ elevation at $x$ } ) \\
    g_2(x) &= ( \text{ scrub cover at $x$ } ) \\
    \vdots
  \end{align*}
  and define
  \begin{align*}
    \text{jump rate at $x$:} \qquad u(x) &= 1/\left(1+\exp\left( - {\color{blue} \sum_{k=1}^n \alpha_k g_k(x) } \right) \right) \\
    \text{habitat quality at $x$:} \qquad \rho(x) &= e^\gamma / \left(1+\exp\left( - {\color{blue} \sum_{k=1}^n \beta_k g_k(x) } \right) \right)
  \end{align*}

  Then choose $\alpha$, $\beta$, and $\gamma$ so that
  \[
  dX_t = \rho(X_t) \nabla u(X_t) dt + \sqrt{\rho(X_t) u(X_t)} dB_t  
  \]
  fits the data.
  % {\aside $X_t$ is a {\newthing inhomogeneous diffusion in a potential}}

\end{frame}


%%%%%
\begin{frame}{The rest of the method}
  \begin{columns}
    \begin{column}{0.5\textwidth}
      \begin{itemize}

        \item<1-> Sequence divergence \\
          $\approx$ {\newthing mean hitting times} of a lineage
          \vspace{2em}

        \item<2-> fit parameters by solving PDE 
          % \\ {\aside (multigrid methods)}

        % \item<2-> as are {\newthing derivatives}
        %   with respect to the parameters
           \vspace{2em}

        \item<3-> {\newthing Results:} quantitative comparison of different development scenarios.

      \end{itemize}
    \end{column}
    \begin{column}{0.5\textwidth}
      \centering
      \includegraphics<1>[height=.9\textheight]{\basedir/resistance-approx-right}
      % \only<2>{
      %   \begin{align*}
      %     \rho(x) \nabla \cdot ( u(x) \nabla h_A(x) ) &= -1  \\
      %     \qquad \text{for } x \notin &A  \\
      %     h_A(x) &= 0  \\
      %      \qquad \text{for } x \in &A   
      %   \end{align*}
      % }
      \only<2>{
      \vspace{-.4in}

      \includegraphics[height=\textheight]{\basedir/example-hitting-times}
      }
      \only<3>{
        % \includegraphics[width=0.8\textwidth]{\basedir/tort-times-alt-1.png}
        % \includegraphics[width=0.8\textwidth]{\basedir/tort-times-alt-2.png}
        \includegraphics[width=0.8\textwidth]{\basedir/tort-times-alt-2-cropped.png}
      }
    \end{column}
  \end{columns}
\end{frame}


%%%%%%%
\begin{frame}{Slow and steady}

    Best-fit model uses only the shape of the range {\struct (flat!)},

    but better fit should be possible.

    \begin{overlayarea}{\textwidth}{0.8\textheight}
        \centering
        \vfill

            % Lake Manix; Mojave River
        % \includegraphics[height=0.8\textheight]{\basedir/glaciallakes.png}
        \includegraphics<2>[height=0.8\textheight]{\basedir/lakemap4.png}

    \end{overlayarea}

\end{frame}


%%%%%%%%% %%%%%%%%%%%% %%%%%%%%%%%%
\section{Next steps}

%%%%%%%%
\begin{frame}{What's next}

  \begin{itemize}

    \item
      Use haplotype sharing
      to fit models of lineage movement,
      \pause

    \item
      including time-inhomogeneous models.
      \pause

  \end{itemize}

  \vspace{2em}

  Putting it all together,
  \begin{itemize}

      \item
        spatially explicit models
          \pause
      
      \item 
          of explicit regulatory networks
          \pause

      \item 
          adapting to heterogeneous selective pressures.

  \end{itemize}

\end{frame}


\begin{frame}

  \begin{center}
    {\Large Thanks! }



  \begin{columns}[c]
    \begin{column}{.4\textwidth}
      \begin{center}
          {Evan McCartney-Melstad}

          \includegraphics[width=\textwidth]{\basedir/evan}

  {Graham Coop}

  \includegraphics[width=\textwidth]{\basedir/coop_graham_full1}

      \end{center}
    \end{column}
    \begin{column}{.6\textwidth}

        \vspace{-2em}
      {Brad Shaffer}

        \includegraphics[width=0.75\textwidth]{\basedir/brad}

      Gideon Bradburd
      \vspace{0.5em}

      Jannet Vu
      \vspace{0.5em}

      \textbf{Data:}
      POPRES
      //
      Fran Sandmeier, Chava Weitzman, Dick Tracy
      \vspace{0.5em}

      \textbf{Funding:}

      NSF: ABI // 
      Sloan Foundation

    \end{column}
  \end{columns}


  \end{center}
\end{frame}

%%%%%%%%%% %%%%%%%%%%%%% %%%%%%%%%%


\section*{More details about fitting hitting times}

%%%%%
\begin{frame}{Pairwise divergence and hitting times}
      \includegraphics<1>[width=\textwidth]{\basedir/lineages-hitting-time-coal}
      \includegraphics<2>[width=\textwidth]{\basedir/lineages-hitting-time-divergence}
      \includegraphics<3>[width=\textwidth]{\basedir/lineages-hitting-time-divergence-decomp}
      % \includegraphics<4>[width=\textwidth]{\basedir/lineages-hitting-time-divergence-notation}
\end{frame}

%%%%%%%
\begin{frame}{A more tractable problem}
  \begin{center}
      \includegraphics[width=\textwidth]{\basedir/lineages-hitting-time-resistance-coal}
  \end{center}
  Replace {\oldthing $\tau_N$} by commute time \\
  (a.k.a.\ {\newthing resistance distance}, McRae et al)
  \begin{align*}
    N(i,j) &= \E[ \text{( time to get near $j$ started from $i$ )} ] \\
    H(i,j) &\approx \frac{ N(i,j) + N(j,i) }{2} .
  \end{align*}
\end{frame}

%%%%%%%
\begin{frame}{A more tractable problem}
  \begin{center}
      \includegraphics[width=\textwidth]{\basedir/lineages-hitting-time-resistance-divergence}
  \end{center}
  Find parameters $\alpha$, $\beta$, $\gamma$, $T$ to minimize
  \[
  \sum_{ij} | D(i,j) - T - H(i,j) |^2
  \]
\end{frame}

%%%%%%%
\begin{frame}{Fitting hitting times}

  Parameters $\alpha$, $\beta$, $\gamma$ determine $u$ and $\rho$:
  \[
  Gf(x) := \rho(x) \nabla \cdot ( u(x) \nabla f(x) ) .
  \]
  Then
  \[
  h_A(x) := \E[ \text{ time for $X_t$ to hit $A$ from $x$ } ] ,
  \]
  solves
  \begin{align*}
    G h_A(x) &= -1  \qquad \text{for } x \notin A \\
    h_A(x) &= 0  \qquad \text{for } x \in A 
  \end{align*}
  {\struct Keywords:} multigrid methods for elliptic PDE.
  

  \ldots and we can get derivatives by solving the same sort of equation:
  \begin{align*}
    G (\partial_\alpha h_A(x)) &= -(\partial_\alpha G) h_A(x)  \\
    G (\partial_\alpha^2 h_A(x)) &= -(\partial_\alpha^2 G) h_A(x) -2 (\partial_\alpha G)(\partial_\alpha h_A(x))
  \end{align*}
  \ldots and use a \alert{trust region algorithm} to optimize.

\end{frame}

%%%%%
\begin{frame}{}
  \fullslide{
  \centering
  \includegraphics[height=\textheight]{\basedir/example-hitting-times}
  }
\end{frame}


\end{document}
